\begin{SCn}
	\scnsectionheader{Предметная область и онтология интеллектуальных компьютерных систем нового поколения}
	\begin{scnsubstruct}

		\begin{scnrelfromlist}{дочерний раздел}
			\scnitem{Предметная область и онтология смыслового представления информации}
			\scnitem{Предметная область и онтология многоагентных моделей решателей задач, основанных на смысловом представлении информации}
			\scnitem{Предметная область и онтология онтологических моделей интерфейсов интеллектуальных компьютерных систем, основанных на смысловом представлении информации}
		\end{scnrelfromlist}

		\scntext{аннотация}{Данный раздел и дочерние ему разделы являются
			уточнением и обоснованием наших предложений, направленных на построение
			компьютерных систем следующего поколения, основанных на смысловом представлении
			обрабатываемой информации. 
			\\В предметной области рассмотрены принципы построения интеллектуальных компьютерных систем нового поколения. В
			качестве ключевых свойств интеллектуальных систем нового поколения выделяются их самообучаемость,
			интероперабельность и семантическая совместимость. Рассматривается подход к обеспечению ука
			занных свойств на основе смыслового представления информации и многоагентных моделей обработки
			информации.}
		\scntext{основной тезис}{Для \uline{любой} \textit{компьютерной
				системы} можно построить эквивалентную ей логико-семантическую модель,
			основанную на смысловом представлении обрабатываемой информации}

			\begin{scnrelfromlist}{ключевой знак}
				\scnitem{Технология OSTIS}
				\scnitem{УСК}
			\end{scnrelfromlist}

			\begin{scnrelfromlist}{ключевое понятие}
				\scnitem{интеллектуальная компьютерная система нового поколения}
				\scnitem{интероперабельная интеллектуальная компьютерная систем}
				\scnitem{самообучаемая интеллектуальная компьютерная система}
				\scnitem{семантическая сеть}
				\scnitem{многоагентная система обработки информации в общей памяти}
			\end{scnrelfromlist}

		\scnheader{логико-семантическая модель компьютерной системы}
		\scntext{пояснение}{Главным фактором обеспечения совместимости
			различных видов знаний, различных моделей решения задач и различных
			компьютерных систем в целом является
			\begin{scnitemize}
				\item унификация (стандартизация) представления информации в памяти
				компьютерных систем;
				\item унификация принципов организации обработки информации в памяти
				компьютерных систем.
			\end{scnitemize}
			Унификация представления информации, используемой в компьютерных
			системах, предполагает:
			\begin{scnitemize}
				\item синтаксическую унификацию используемой информации  унификацию
				формы представления (кодирования) этой информации. При этом следует отличать:
				\begin{scnitemizeii}
					\item кодирование информации в памяти компьютерной системы (внутреннее
					представление информации);
					\item внешнее представление информации, обеспечивающее однозначность
					интерпретации (понимания, трактовки) этой информации разными пользователями и
					разными компьютерными системами;
				\end{scnitemizeii}
				\item семантическую унификацию используемой информации, в основе
				которой лежит согласование и точная спецификация всех (!) используемых понятий
				(концептов) с помощью иерархической системы формальных онтологий.
			\end{scnitemize}}

		\scnheader{стандарт}
		\scnhaselement{Стандарт OSTIS}
		\begin{scnindent}
			\scnidtf{Предлагаемый нами стандарт логико-семантических моделей
				компьютерных систем,  основанных на смысловом представлении информации, и
				технологии разработки таких моделей и соответствующих компьютерных систем}
		\end{scnindent}
		\scnidtf{знания о структуре и принципах функционирования искусственных
			систем соответствующего класса}
		\scnidtf{онтология искусственных систем некоторого класса}
		\scnidtf{теория искусственных систем некоторого класса}
		\scntext{пояснение}{Важно отметить, что грамотная унификация
			(стандартизация) должна не ограничивать творческую свободу разработчика, а
			гарантировать \uline{совместимость} его результатов с результатами других
			разработчиков. Подчеркнем также, что текущая версия любого \textit{стандарта}
			-- это не догма, а только опора для дальнейшего его совершенствования.Целью
			качественного \textit{стандарта} является не только обеспечения совместимости
			технических решений, но и минимизация дублирования (повторения) таких решений.
			Один из важных критериев качества \textit{стандарта} --- ничего
			лишнего.\textit{Стандарты}, как и другие важные для человечества
			\textit{знания}, должны быть формализованы и должны постоянно
			совершенствоваться с помощью специальных \textit{интеллектуальных компьютерных
				систем}, поддерживающих процесс эволюции стандартов путем согласования
			различных точек зрения.}
			
		\scnheader{семантическая совместимость компьютерных систем}
		\scntext{пояснение}{Уровень совместимости \textit{компьютерных
				систем} определяется трудоемкостью реализации процедур интеграции (объединения,
			соединения знаний этих систем), а также трудоемкостью и глубиной интеграции
			входящих в эти системы \textit{решателей задач} (интеграции навыков и
			интерпретаторов этих навыков). Подчеркнем при этом, что интеграция интеграции
			рознь --- от эклектики до гибридности и синергетичности дистанция огромного
			размера.
			
			Совместимые \textit{компьютерные системы} --- это компьютерные системы,
			для которых существует автоматически выполняемая процедура их интеграции,
			превращающая эти системы в единую \textit{гибридную систему}, в рамках которой
			каждая интегрируемая компьютерная система в процессе своего функционирования
			может свободно использовать любые необходимые информационные ресурсы (знания и
			навыки), входящие в состав другой интегрируемой компьютерной системы.
			
			Целостную \textit{компьютерную систему} можно рассматривать как решатель задач,
			интегрировавший несколько моделей решения задач и обладающий средствами
			взаимодействия с внешней для себя средой (с другими компьютерными системами, с
			пользователями).
			
			Таким образом, для того, чтобы повысить уровень совместимости
			\textit{компьютерных систем}, необходимо преобразовать их к виду
			\textit{многоагентных систем}, работающих над общей семантической памятью.
			Такие \textit{компьютерные системы} не всегда целесообразно непосредственно
			объединять (интегрировать) в более крупные \textit{компьютерные системы}.
			Иногда целесообразнее их объединять в \textit{коллективы взаимодействующих
			компьютерных систем}. Но при создании таких коллективов компьютерных систем
			унификация и совместимость таких систем также очень важны, т.к. существенно
			упрощают обеспечение высокого уровня их взаимопонимания. Так, например,
			противоречия между компьютерными системами, входящими в коллектив, можно
			обнаруживать путем анализа на непротиворечивость \textit{виртуальной
			объединенной базы знаний} этого коллектива. Более того, непротиворечивость
			указанной виртуальной базы знаний можно считать одним из критериев
			семантической совместимости систем, входящих в соответствующий
			коллектив.}
			
		\scnheader{компьютерная система, основанная на смысловом представлении информации}
		\scntext{пояснение}{Предлагаемое нами устранение проблем современных
			информационных технологий путем перехода к \textit{смысловому представлению
				информации} в памяти компьютерных систем фактически преобразует современные
			компьютерные системы (в том числе и современные интеллектуальные компьютерные
			системы) в \textit{компьютерные системы, основанные на смысловом представлении
				информации}, которые являются не альтернативной ветвью развития
			\textit{компьютерных систем}, а естественным этапом их эволюции, направленным
			на обеспечение высокого уровня их \textit{обучаемости} и, в первую очередь,
			\textit{совместимости}.
			
			Архитектура \textit{компьютерных систем, основанных на
			смысловом представлении информации} (см. \textit{Рис. Архитектура компьютерных
			систем, основанных на смысловом представлении информации}) практически
			совпадает с архитектурой \textit{интеллектуальных компьютерных систем},
			основанных на знаниях. Отличие здесь заключаются в том, что в
			\textit{компьютерных системах, основанных на смысловом представлении
			информации}:
			\begin{scnitemize}
				\item база знаний имеет смысловое представление;
				\item интерпретатор знаний и навыков представляет собой коллектив
				\textit{агентов}, осуществляющих обработку \textit{базы знаний}.
			\end{scnitemize}
			Как следствие этого, \textit{компьютерные системы, основанная на
			смысловом представлении информации}, обладают высоким уровнем
			\textit{обучаемости}, т.е. способностью быстро приобретать новые и
			совершенствовать уже приобретенные знания и навыки и при этом не иметь никаких
			ограничений на вид приобретаемых и совершенствуемых ею знаний и навыков, а
			также на их совместное использование.
			
			Более того, при согласовании соответствующих стандартов, а также при перманентном совершенствовании этих
			стандартов и при грамотной их поддержке в условиях интенсивной эволюции как
			самих стандартов, так и \textit{компьютерных систем, основанных на смысловом
			представлении информации} (речь идет о постоянной поддержке соответствия между
			текущим состоянием компьютерных систем и текущим состоянием эволюционируемых
			стандартов), \textit{компьютерные системы, основанные на смысловом
			представлении информации} и их компоненты обладают весьма высокой степенью
			\textit{совместимости}.
			
			Это, в свою очередь, практически исключает дублирование
			инженерных решений и дает возможность существенно ускорить разработку
			\textit{компьютерных систем, основанных на смысловом представлении информации}
			с помощью постоянно расширяемой библиотеки многократно используемых и
			совместимых между собой компонентов. 
			
			Основным лейтмотивом перехода от современных компьютерных систем (в том числе интеллектуальных) к
			\textit{компьютерным системам, основанным на смысловом представлении
				информации}, хранимой в ее памяти, является создание \textbf{\textit{общей
					семантической теории компьютерных систем}}, включающей в себя:
			\begin{scnitemize}
				\item cемантическую теорию \textit{знаний} и \textit{баз знаний};
				\item семантическую теорию \textit{задач} и \textit{моделей решения
					задач};
				\item cемантическую теорию \textit{взаимодействия информационных
					процессов};
				\item cемантическую теорию пользовательских и, в том числе,
				естественно-языковых интерфейсов;
				\item cемантическую теорию невербальных (сенсорно-эффекторных)
				интерфейсов;
				\item теорию универсальных интерпретаторов \textit{логико-семантических
					моделей компьютерных систем} и, в частности, теорию семантических компьютеров.
			\end{scnitemize}
			Эпицентром следующего этапа развития информационных технологий является
			решение проблемы обеспечения \textbf{\textit{семантической совместимости}}
			\textit{компьютерных систем} и их компонентов. Для решения этой проблемы
			необходим
			\begin{scnitemize}
				\item переход от традиционных компьютерных систем и от современных
				интеллектуальных компьютерных систем к \textit{компьютерным системам,
					основанным на смысловом представлении информации};
				\item разработка \textit{стандарта компьютерных систем, основанных на
					смысловом представлении информации}.
			\end{scnitemize}
			Очевидно, что \textit{компьютерные системы, основанных на смысловом
				представлении информации} являются компьютерными системами нового поколения,
			устраняющие многие недостатки современных компьютерных систем. Но для массовой
			разработки таких систем необходима соответствующая технология, которая должна
			включать в себя
			\begin{scnitemize}
				\item теорию \textit{компьютерных систем, основанных на смысловом
					представлении информации} и комплекс всех стандартов, обеспечивающих
				совместимость разрабатываемых систем;
				\item методы и средства проектирования \textit{компьютерных систем,
					основанных на смысловом представлении информации};
				\item методы и средства перманентного совершенствования самой
				технологии.
			\end{scnitemize}}
			\begin{scnindent}
				\scnrelfrom{иллюстрация}{\scnfileimage{Contents/part_intro/src/images/arch.pdf}}
					\begin{scnindent}
						\scnidtf{Рис. Архитектура компьютерных систем, \textit{основанных на смысловом представлении информации}}
					\end{scnindent}
			\end{scnindent}
		\bigskip

		\scnheader{уровень интеллекта индивидуальных интеллектуальных систем}
		\scntext{примечание}{Важнейшим направлением повышения уровня интеллекта индивидуальных интеллектуальных кибернетических
			систем является переход к коллективам индивидуальных интеллектуальных кибернетических систем и далее к
			иерархическим коллективам интеллектуальных кибернетических систем, членами которых являются как 
			индивидуальные интеллектуальные кибернетические системы, так и коллективы индивидуальных интеллектуальных
			кибернетических систем, а также иерархические коллективы интеллектуальных кибернетических систем.}
		\scntext{примечание}{Аналогичным образом необходимо повышать уровень интеллекта и индивидуальных интеллектуальных \uline{компьютерных}
			систем (искусственных кибернетических систем). Но при этом надо помнить, что далеко не каждое объединение
			\uline{интеллектуальных} кибернетических систем (в том числе и компьютерных систем) становится интеллектуальным
			коллективом. Для этого необходимо соблюдение дополнительных требований, предъявляемых \uline{ко всем} членам
			интеллектуальных коллективов. Важнейшим из них является требование высокого уровня интероперабельности,
			то есть способности к эффективному взаимодействию с другими членами коллектива. Переход от современных 
			интеллектуальных компьютерных систем к интероперабельным интеллектуальным компьютерным системам является
			ключевым фактором перехода к интеллектуальным компьютерным системам нового поколения, обеспечивающим
			существенное повышение уровня автоматизации человеческой деятельности.}

		\scnheader{комплексная задача}
		\scnidtf{complex problem}
		\scnidtf{задача, решение которой невозможно с помощью одной модели решения задач, одного вида знаний, одной интеллектуальной компьютерной системы}
		\scntext{примечание}{Расширение областей применения интеллектуальных компьютерных систем требует перехода к решению \uline{комплексных} задач.}
		\scntext{примечание}{Решение:
		\begin{scnitemize}
			\item Переход к гибридным индивидуальным интеллектуальным компьютерным системам, в которых осуществляется конвергенция и интеграция различных моделей решения задач и различных видов знаний.
			\item Переход к \uline{коллективам} семантически совместимых самостоятельных интеллектуальных компьютерных систем, в которых обеспечивается:
			\begin{scnitemizeii}
				\item интероперабельность объединяемых интеллектуальных компьютерных систем
				\item конвергенция объединяемых интеллектуальных компьютерных систем при сохранении их самостоятельности.
			\end{scnitemizeii}
		\end{scnitemize}}

		\input{Contents/part_intro/src/sem_systems/segment1.tex}
		\scnsegmentheader{Принципы, лежащие в основе онтологических моделей мультимодальных интерфейсов интеллектуальных компьютерных систем нового поколения}

\begin{scnsubstruct}
    \begin{scnrelfromlist}{ключевое понятие}
    	\scnitem{смысловая память}
    	\scnitem{графодинамическая память}
    	\scnitem{ассоциативная память с информационным доступом по образцу произвольного размера и
            конфигурации}
        \scnitem{система ситуационного децентрализованного управления информационными процессами}
    	\scnitem{многоагентная система обработки информации в общей памяти}
    	\scnitem{язык смыслового представления задач}
        \scnitem{универсальный язык смыслового представления знаний}
    	\scnitem{язык смыслового представления методов}
        \begin{scnindent}
    		\scnidtf{интегрированный язык смыслового представления различного вида программ}
    	\end{scnindent}
    	\scnitem{инсерционная программа}
    \end{scnrelfromlist}
   
    \begin{scnrelfromlist}{ключевое знание}
    	\scnitem{Принципы, лежащие в основе решателей задач индивидуальных интеллектуальных компьютерных
            систем нового поколения}
    \end{scnrelfromlist}

    \scnheader{решатель задач интеллектуальных компьютерных систем нового поколения}
    \begin{scnrelfromlist}{предъявляемые требования}
        \scnfileitem{решатель задач интеллектуальных компьютерных систем нового поколения должен уметь решать 
            интеллектуальные задачи}
            \begin{scnrelfromlist}{виды задач}
                \scnfileitem{некачественно сформулированная задача}
                \begin{scnindent}
                    \scnidtf{задача, формулировка которой содержит различные не-факторы (неполнота, нечеткость,
                        противоречивость (некорректность) и так далее)}
                \end{scnindent}
                \scnfileitem{задача, для решения которой, кроме самой формулировки задачи и соответствующего метода ее
                    решения необходима дополнительная, но априори неизвестно какая информация об объектах, указанных
                    в формулировке (постановке) задачи. При этом указанная дополнительная информация
                    может присутствовать, а может и отсутствовать в текущем состоянии базы знаний интеллектуальных
                    компьютерных систем. Кроме того, для некоторых задач может быть задана (указана) та область
                    базы знаний, использования которой достаточно для поиска или генерации (в частности, логического
                    вывода) указанной дополнительной требуемой информации. Такую область базы знаний будем
                    называть областью решения соответствующей задачи}
                \scnfileitem{задача, для которой соответствующий метод ее решения в текущий момент не известен}
                \begin{scnrelfromlist}{решение}
                    \scnfileitem{переформулировать задачу, то есть сгенерировать (логически вывести) логически эквивалентную
                        формулировку исходной задачи, для которой метод ее решения в текущий момент является
                        известным}
                    \scnfileitem{свести исходную задачу к семейству подзадач, для которых методы их решения в текущий
                        момент известны.}
                \end{scnrelfromlist}
            \end{scnrelfromlist}
        \scnfileitem{процесс решения задач в интеллектуальных компьютерных системах нового поколения реализуется коллективом
            информационных агентов, обрабатывающих базу знаний интеллектуальных компьютерных систем}
        \scnfileitem{управление информационными процессами в памяти интеллектуальных компьютерных систем нового
            поколения осуществляется децентрализованным образом по принципам ситуационного управления}
    \end{scnrelfromlist}

    \scnheader{ситуационное управление}
    \scnidtf{situational management}
    \scnidtf{ситуационно-событийное управление}
    \scntext{пояснение}{управление последовательностью выполнения действий, при котором условием (\scnqq{триггером}) инициирования
        указанных действий является:
        \begin{scnitemize}
            \item возникновение некоторых ситуаций (условий, состояний);
            \item и/или возникновение некоторых событий.
        \end{scnitemize}}
        
    \scnheader{ситуация}
    \scnidtf{situation}
    \scnidtf{структура, описывающая некоторую временно существующую конфигурацию связей между некоторыми
        сущностями}
    \scnidtf{описание временно существующего состояния некоторого фрагмента (некоторой части) некоторо
        динамической системы}

    \scnheader{событие}
    \scnidtf{event}
    \scnsuperset{возникновение временной сущности}
    \begin{scnindent}
        \scnidtf{появление, рождение, начало существования некоторой временной сущности}
    \end{scnindent}
    \scnsuperset{исчезновение временной сущности}
    \begin{scnindent}
        \scnidtf{прекращение, завершение существования некоторой временной сущности}
    \end{scnindent}
    \scnsuperset{переход от одной ситуации к другой}
    \begin{scnindent}
        \scntext{примечание}{Здесь учитывается не только факт возникновения новой ситуации, но и ее предыстория --- то есть та
        ситуация, которая ей непосредственно предшествует. Так, например, реагируя на аномальное значение
        какого-либо параметра, нам важно знать:
        \begin{scnitemize}
            \item какова динамика изменения этого параметра (увеличивается он или уменьшается и с какой скоростью);
            \item какие меры были предприняты ранее для ликвидации этой аномалии.
        \end{scnitemize}}
    \end{scnindent}

    \scnheader{решатель задач индивидуальной интеллектуальной компьютерной системы нового поколения}
    \begin{scnrelfromlist}{принципы, лежащие в основе}
        \scnfileitem{смысловое представление обрабатываемых знаний}
        \scnfileitem{семантически неограниченный ассоциативный доступ к различным фрагментам знаний, хранимым в
            памяти интеллектуальных компьютерных систем нового поколения (доступ по заданному образцу произвольного
            размера и произвольной конфигурации)}
        \scnfileitem{графодинамический характер обработки знаний в памяти, при котором обработка знаний сводится не
            только к изменению состояния атомарных фрагментов (ячеек) памяти, но и к изменению конфигурации
            связей между этими атомарными фрагментами}
        \scnfileitem{ситуационное децентрализованное управление процессом обработки знаний, а также процессом организации
            взаимодействия интеллектуальных компьютерных систем с внешней средой}
        \scnfileitem{использование семантически мощного языка задач, обеспечивающего представление формулировок самых
            различных задач, которые могут решаться либо в рамках памяти интеллектуальной компьютерной
            системы, либо во внешней среде и которые осуществляют инициирование соответствующих процессов
            решения задач}
        \scnfileitem{многоагентный характер реализации процессов решения инициированных задач, в основе которого лежит
            иерархическая система агентов, каждый из которых активизируются при возникновении в памяти
            интеллектуальной компьютерной системы соответствующий ситуации или соответствующего события}
    \end{scnrelfromlist}
\end{scnsubstruct}

		\input{Contents/part_intro/src/sem_systems/segment3.tex}
		\input{Contents/part_intro/src/sem_systems/segment4.tex}
	\end{scnsubstruct}
\end{SCn}
\scnsourcecomment{Завершили Раздел \scnqqi{Предметная область и онтология интеллектуальных компьютерных систем нового поколения}}
