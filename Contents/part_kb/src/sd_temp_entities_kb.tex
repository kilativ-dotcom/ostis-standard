\begin{SCn}
    \scnsectionheader{Предметная область и онтология ситуаций и событий, описывающих динамику баз знаний ostis-систем}
    \begin{scnsubstruct}
        \scntext{введение}{Обработка информации в \textit{sc-памяти} (т.е. динамика базы знаний, хранимой в \textit{sc-памяти}) в конечном счете сводится:
            \begin{scnitemize}
                \item к появлению в \textit{sc-памяти} новых актуальных \textit{sc-узлов} и \textit{sc-коннекторов};
                \item к логическому удалению актуальных \textit{sc-элементов}, т.е. к переводу их в неактуальное состояние (это необходимо для хранения протокола изменения состояния базы знаний, в рамках которого могут описываться действия по удалению \textit{sc-элементов});
                \item к возврату логически удаленных \textit{sс-элементов} в статус актуальных (необходимость в этом может возникнуть при откате базы знаний к какой-нибудь ее прошлой версии);
                \item к физическому удалению \textit{sc-элементов};
                \item к изменению состояния актуальных (логически не удаленных \textit{sc-элементов}) --- \textit{sc-узел} может превратиться в \textit{sc-ребро}, \textit{sc-ребро} может превратиться в \textit{sc-дугу}, \textit{sc-дуга} может поменять направленность, \textit{sc-дуга} общего вида может превратиться в \textit{константную стационарную sc-дугу принадлежности}, и т.д.;
            \end{scnitemize}
            Подчеркнем, что временный характер самого \textit{sc-элемента} (т.к. он может появиться или исчезнуть) никак не связан с возможно временным характером сущности, обозначаемой этим \textit{sc-элементом}. Т.е. временный характер самого sc-элемента и временный характер сущности, которую он обозначает --- абсолютно разные вещи.\\
            Таким образом, следует четко отличать динамику внешнего мира, описываемого базой знаний, а динамику самой базы знаний (динамику внутреннего мира). При этом очень важно, чтобы описание динамики базы знаний также входило в состав каждой базы знаний.\\
            К числу понятий, используемых для описания динамики базы знаний относятся:
            \begin{scnitemize}
                \item логически удаленный sc-элемент;
                \item сформированное множество;
                \item вычисленное число;
                \item сформированное высказывание.
            \end{scnitemize}}
        
        \scnheader{Предметная область ситуаций и событий, описывающих динамику баз знаний ostis-систем}
        \scnidtf{Предметная область, описывающая динамику базы знаний, хранимой в sc-памяти}
        \scniselement{предметная область}
        \begin{scnhaselementrole}{максимальный класс объектов исследования}
            {ситуация}
        \end{scnhaselementrole}
        \begin{scnhaselementrolelist}{класс объектов исследования}
            \scnitem{sc-элемент}
            \scnitem{наcтоящий sc-элемент}
            \scnitem{логически удаленный sc-элемент}
            \scnitem{число}
            \scnitem{невычисленное число}
            \scnitem{вычисленное число}
            \scnitem{понятие}
            \scnitem{основное понятие}
            \scnitem{неосновное понятие}
            \scnitem{понятие, переходящее из основного в неосновное}
            \scnitem{понятие, переходящее из неосновного в основное}
            \scnitem{специфицированная сущность}
            \scnitem{недостаточно специфицированная сущность}
            \scnitem{достаточно специфицированная сущность}
            \scnitem{средне специфицированная сущность}
            \scnitem{структура}
            \scnitem{файл}
            \scnitem{событие в sc-памяти*}
            \scnitem{элементарное событие в sc-памяти*}
            \scnitem{событие добавления sc-дуги, выходящей из заданного sc-элемента*}
            \scnitem{событие добавления sc-дуги, входящей в заданный sc-элемент*}
            \scnitem{событие добавления sc-ребра, инцидентного заданному sc-элементу*}
            \scnitem{событие удаления sc-дуги, выходящей из заданного sc-элемента*}
            \scnitem{событие удаления sc-дуги, входящей в заданный sc-элемент*}
            \scnitem{событие удаления sc-ребра, инцидентного заданному sc-элементу*}
            \scnitem{событие удаления sc-элемента*}
            \scnitem{событие изменения содержимого файла*}
        \end{scnhaselementrolelist}
        
        \scnheader{sc-элемент}
        \scnidtf{sc-element}
        \begin{scnreltoset}{разбиение}
            \scnitem{наcтоящий sc-элемент}
            \scnitem{логически удаленный sc-элемент}
        \end{scnreltoset}
        
        \scnheader{наcтоящий sc-элемент}
        \scniselement{ситуативное множество}
        
        \scnheader{логически удаленный sc-элемент}
        \scniselement{ситуативное множество}
        
        \scnheader{число}
        \begin{scnsubdividing}
            \scnitem{невычисленное число}
            \scnitem{вычисленное число}
        \end{scnsubdividing}
        
        \scnheader{невычисленное число}
        \scniselement{ситуативное множество}

        \scnheader{вычисленное число}
        \scniselement{неситуативное множество}
        
        \scnheader{понятие}
        \begin{scnsubdividing}
            \scnitem{основное понятие}
            \scnitem{неосновное понятие}
            \scnitem{понятие, переходящее из основного в неосновное}
            \scnitem{понятие, переходящее из неосновного в основное}
        \end{scnsubdividing}
        
        \scnheader{основное понятие}
        \scnidtf{основное понятие для данной ostis-системы}
        \scniselement{ситуативное множество}
        \scntext{пояснение}{К \textbf{\textit{основным понятиям}} относятся те понятия, которые активно используются в системе и могут быть ключевыми элементами sc-агентов. К \textbf{\textit{основным понятиям}} относятся также все неопределяемые понятия.}
        
        \scnheader{неосновное понятие}
        \scnidtf{non-basic concept}
        \scnidtf{дополнительное понятие}
        \scnidtf{вспомогательное понятие}
        \scnidtf{неосновное понятие для данной ostis-системы}
        \scniselement{ситуативное множество}
        \scntext{пояснение}{Каждое \textbf{\textit{неосновное понятие}} должно быть строго определено на основе \textit{основных понятий}. Такие \textbf{\textit{неосновные понятия}} используются только для понимания и правильного восприятия вводимой информации, в том числе, для выравнивания онтологий. Ключевым элементом \textit{sc-агентов} \textbf{\textit{неосновные понятия}} быть не могут.}
        \scntext{правило идентификации экземпляров}{В случае, когда некоторое понятие полностью перешло из \textit{основных понятий} в неосновные, то есть стало \textbf{\textit{неосновным понятием}}, и соответствующее ему \textit{основное понятие} (через которое оно определяется) в рамках некоторого внешнего языка имеет одинаковый с ним основной идентификатор, то к идентификатору \textbf{\textit{неосновного понятия}} спереди добавляется знак \#. Если при этом соответствуюшее \textit{основное понятие} имеет в идентификаторе знак \$, добавленный в процессе перехода, то этот знак удаляется. Если указанные понятия имеют разные основные идентификаторы в рамках этого внешнего языка, то никаких дополнительных средств идентификации не используется.\\
        	Например:\\
            \textit{\#трансляция sc-текста}\\
            \textit{\#scp-программа}}
            
        \scnheader{понятие, переходящее из основного в неосновное}
        \scniselement{ситуативное множество}
        
        \scnheader{понятие, переходящее из неосновного в основное}
        \scniselement{ситуативное множество}
        \scntext{правило идентификации экземпляров}{В случае, когда текущее \textit{основное понятие} и соответствующее ему \textbf{\textit{понятие, переходящее из неосновного в основное}} в рамках некоторого внешнего языка имеют одинаковый основной идентификатор, то к идентификатору понятия, переходящего из неосновного в основное спереди добавляется знак \$. Если указанные понятия имеют разные основные идентификаторы в рамках этого внешнего языка, то никаких дополнительных средств идентификации не используется.\\
        	Например:\\
            \textit{\$трансляция sc-текста}\\
            \textit{\$scp-программа}}
            
        \scnheader{специфицированная сущность}
        \begin{scnsubdividing}
            \scnitem{недостаточно специфицированная сущность}
            \scnitem{достаточно специфицированная сущность}
            \scnitem{средне специфицированная сущность}
        \end{scnsubdividing}
        
        \scnheader{достаточно специфицированная сущность}
        \scntext{пояснение}{К \textbf{\textit{достаточно специфицированным сущностям}} предъявляются следующие требования:
            \begin{scnitemize}
                \item Если сущность не является понятием, то для нее должны быть указаны
                \begin{scnitemizeii}
                    \item различные варианты обозначающих ее внешних знаков;
                    \item классы, которым она принадлежит;
                    \item связки, которыми она связана с другими сущностями (с указанием соответствующего отношения);
                    \item значения параметров, которыми она обладает;
                    \item те разделы базы знаний, в которых указанная сущность является ключевой;
                    \item предметные области, в которые данная сущность входит.
                \end{scnitemizeii}
                \item Если специфицированная сущность является понятием, то для нее должны быть указаны:
                \begin{scnitemizeii}
                    \item различные варианты внешних обозначений этого понятия;
                    \item предметные области, в которых это понятие исследуется;
                    \item определение понятия;
                    \item пояснения;
                    \item разделы базы знаний, в которых это понятие является ключевым;
                    \item описание примера --- пример экземпляра понятия.
                \end{scnitemizeii}
            \end{scnitemize}}
        
        \scnheader{структура}
        \begin{scnsubdividing}
            \scnitem{сформированная структура}
            \scnitem{несформированная структура}
        \end{scnsubdividing}
        \begin{scnsubdividing}
            \scnitem{недостаточно сформированная структура}
            \scnitem{достаточно сформированная структура}
            \scnitem{структура, имеющая средний уровень сформированности}
        \end{scnsubdividing}
        
        \scnheader{файл}
        \begin{scnsubdividing}
            \scnitem{недостаточно сформированный внутренний файл}
            \scnitem{достаточно сформированный внутренний файл}
            \scnitem{внутренний файл, имеющий средний уровень сформированности}
        \end{scnsubdividing}
        
        \scnheader{событие в sc-памяти}
        \scnsuperset{событие}
        
        \scnheader{элементарное событие в sc-памяти}
        \scnsubset{событие в sc-памяти}
        \scntext{пояснение}{Под \textbf{\textit{элементарным событием в sc-памяти}} понимается такое \textit{событие}, в результате выполнения которого изменяется состояние только одного \textit{sc-элемента}.}
        \begin{scnsubdividing}
            \scnitem{событие добавления sc-дуги, выходящей из заданного sc-элемента}
            \scnitem{событие добавления sc-дуги, входящей в заданный sc-элемент}
            \scnitem{событие добавления sc-ребра, инцидентного заданному sc-элементу}
            \scnitem{событие удаления sc-дуги, выходящей из заданного sc-элемента}
            \scnitem{событие удаления sc-дуги, входящей в заданный sc-элемент}
            \scnitem{событие удаления sc-ребра, инцидентного заданному sc-элементу}
            \scnitem{событие удаления sc-элемента}
            \scnitem{событие изменения содержимого файла}
        \end{scnsubdividing}
        
        \scnheader{точечный процесс}
        \scnidtf{атомарный процесс}
        \scnidtf{условно мгновенный процесс}
        \scnidtf{процесс, длительность которого в данном контексте считается несущественной (пренебрежимо малой)}
        
        \scnheader{элементарный процесс}
        \scnidtf{процесс, детализация которого на входящие в него подпроцессы в текущем контексте не производится}
        \bigskip
    \end{scnsubstruct}
    \scnendcurrentsectioncomment
\end{SCn}
