\begin{SCn}
    \bigskip
    \scnsectionheader{Предметная область и онтология языка внешнего форматированного представления информационных конструкций внутреннего языка ostis-систем}
    \begin{scnstruct}
    	\begin{scnrelfromlist}{ключевой знак}
    		\scnitem{SCn-код}
    		\scnitem{Алфавит SCn-кода\scnsupergroupsign}
    		\scnitem{Синтаксис SCn-кода}
    		\scnitem{Денотационная семантика SCn-кода}
    	\end{scnrelfromlist}
    	\begin{scnrelfromlist}{ключевое понятие}
    		\scnitem{страница sc.n-текста}
    		\scnitem{строка sc.n-текста}
    		\scnitem{линия разметки sc.n-текста}
    		\scnitem{sc.n-предложение}
    		\scnitem{sc.n-элемент}
    		\scnitem{sc.n-коннектор}
    		\scnitem{sc.n-ребро}
    		\scnitem{sc.n-дуга}
    		\scnitem{sc.n-контур}
    		\scnitem{sc.n-рамка}
    	\end{scnrelfromlist}
        \scnheader{SCn-код}
        \scnidtf{SCn-code}
        \scnidtf{Язык структурированного представления знаний \textit{ostis-систем}}
        \scnidtf{Язык внешнего форматированного представления конструкций внутреннего языка \textit{ostis-систем}}
        \scntext{пояснение}{\textit{SCn-код} является языком структурированного внешнего представления текстов \textit{SC-кода} и представляет собой синтаксическое расширение \textit{SCs-кода}, направленное на повышение наглядности и компактности текстов \textit{SCs-кода}.
        	\\SCn-код позволяет перейти от линейных текстов \uline{SCs-кода} к форматированным и фактически двухмерным текстам, в которых появляется декомпозиция исходного линейного текста \uline{SCs-кода} на \uline{строчки}, размещенные по вертикали. При этом начало всех \uline{строчек} текста фиксировано и определяется известным и ограниченным набором правил, что дает возможность использовать это при форматировании \uline{sc.n-текста} (текста, принадлежащего SCn-коду.)}
        \scniselement{язык двухмерных текстов}
        \begin{scnindent}
            \scnidtf{язык, каждый \textit{текст} которого задается (1) множеством входящих в него \textit{символов}, (2) отношением порядка (последовательности) \textit{символов} \scnqq{по горизонтали}, (3) отношением порядка(последовательности) \textit{символов} \scnqq{по вертикали}.}
            \begin{scnindent}
                \scntext{пояснение}{Символ, входящий в состав \textit{двухмерного текста}, в общем случае может иметь четыре \scnqq{соседних} \textit{символа}: (1) \textit{символ}, находящийся от него \uline{слева} в рамках той же \textit{строчки}, (2) \textit{символ}, находящийся от него \uline{справа} в рамках этой же \textit{строчки}, (3) \textit{символ}, находящийся строго \uline{над} ним в предыдущей \textit{строчке} и (4) \textit{символ}, находящийся строго \uline{под ним} в следующей \textit{строчке} текста.}
            \end{scnindent}
        \end{scnindent}
        \scntext{сравнительный анализ}{Благодаря тому, что в состав sc.n-текстов могут входить и sc.s-тексты, и sc.g-тексты (ограниченные sc.n-контуром), SCn-код можно считать интегратором различных внешних языков представления знаний.  Это дает возможность при визуализации и разработке базы знаний ostis-системы недостатки одного из предлагаемых вариантов внешнего представления sc-текстов (SCg-кода, SCs-кода, SCn-кода) компенсировать достоинствами других вариантов.}
        \scntext{примечание}{\textit{SCn-код} предназначен для представления \textit{sc-графов} в виде отформатированных по заданным правилам последовательностей символов, в которых также могут быть использованы базовые средства гипермедиа, такие как графические изображения, а также средства навигации между частями sc.n-текстов. \textit{SCn-код} имеет много общего с \textit{SCs-кодом} и, за исключением некоторых особенностей, является его двумерным форматированным вариантом.}
       
        \begin{scnset}
            \scnitem{SCn-код}
            \scnitem{SCs-код}
        \end{scnset}
        \begin{scnrelfromset}{описание связи}
            \scnitem{SCn-код}
            \begin{scnindent}
                \scnrelfrom{синтаксическое расширение*:синтаксическое ядро языка}{SCs-код}
            \end{scnindent}
        \end{scnrelfromset}
        \scntext{отличие}{Переход от линейности sc.s-текстов к двухмерности sc.n-текстов.}
        \begin{scnindent}
            \scntext{уточнение}{Важной особенностью SCn-кода является \scnqq{двухмерный} характер его текстов. Это проявляется в том, что для каждого фрагмента текста SCn-кода важное значение имеет величина отступа от левого края \textit{строчки}.
            	\\В тексте \textit{SCn-кода} в отличие от текста \textit{SCs-кода} для каждого фрагмента текста важное значение имеет не только то, как этот фрагмент связан с другими фрагментами \scnqq{по горизонтали} (какой фрагмент находится \uline{левее} и какой \uline{правее} на одной и той же \textit{строчке}), но и то, как он связан с другими фрагментами по вертикали(какой фрагмент находится \uline{выше} на предыдущей \textit{строчке} и какой находится \uline{ниже} на следующей \textit{строчке}). Так, например, если в тексте \textit{SCn-кода} некоторый \textit{sc-идентификатор} (внешний идентификатор \textit{sc-элемента}) размещен сразу после вертикальной табуляционной линии и точно \uline{под ним} размещен некоторый \textit{sc.s-коннектор}, то это означает, что указанный \textit{sc-элемент} инцидентен \textit{sc-коннектору}, изображенному указанным \textit{sc.s-коннектором}.
            	\\Для того, чтобы обеспечить точное задание (формулировку) правил двухмерной инцидентности элементов (элементарных фрагментов) sc.n-текстов, вводится понятие \textit{\textbf{страницы} sc.n-текста}, понятие \textit{\textbf{строчки} sc.n-текста}, а также используется специальная \uline{разметка}, представляющая собой вертикальные табуляционные линии, расстояние между которыми примерно равняется максимальной длине sc.s-коннектора (обычно это расстояние равно ширине 4-5 символов).}
        \end{scnindent}
        
        \scnheader{sc.n-текст}
        \scnidtf{sc.n-text}
        \scnidtf{текст SCn-кода}
        \scnidtf{последовательность предложений SCn-кода}
        \scnidtf{последовательность предложений SCn-кода, каждое из которых не является частью какого-либо другого предложения из \uline{этой} последовательности}
        \scntext{примечание}{Каждый \textit{sc.n-текст} может быть представлен в нескольких вариантах идентификации \textit{sc-элементов} представляемого \textit{sc-графа}:
        	\begin{scnitemize}
        		\item с использованием \textit{системных идентификаторов}, которые носят интернациональный характер;
        		\item с использованием \textit{основных идентификаторов*} для русскоязычных пользователей;
        		\item с использованием \textit{основных идентификаторов*} для англоязычных пользователей.
        \end{scnitemize}}
    	\scntext{примечание}{Каждый \textit{sc.n-текст} представляет собой последовательность \textit{sc.n-статей} (аналог форматированного естественно-языкового текста). Каждая \textit{sc.n-статья}, в свою очередь, представляет собой последовательность \textit{sc.n-предложений}, в начале которой помещается заголовок \textit{sc.n-статьи}, который представляет собой идентификатор ключевого \textit{sc-элемента} того \textit{sc-графа}, который представляется данной \textit{sc.n-статьей}. С семантической точки зрения указанный \textit{sc-граф} является семантической окрестностью, центром которой является указанный ключевой sc-элемент. При этом ключевой sc-элемент некоторой статьи обязательно входит в состав каждого из \textit{sc.n-предложений}, но необязательно является компонентом ключевого (самого первого, считая от начала предложения) \textit{sc-коннектора} данного \textit{sc.n-предложения}. Входящие в \textit{sc.n-статью} \textit{sc.n-предложения} являются \textit{sc.s-предложениями} либо некоторыми их модификациями.}
        \begin{scnindent}
            \scnrelfrom{смотрите}{\scncite{IMS}}
        \end{scnindent}
    
        \scnheader{страница sc.n-текста}
        \scnidtf{page of sc.n-text}
        \scnidtf{страница, на которой размещается sc.n-текст}
        \scntext{примечание}{Если sc.n-текст является частью какого-либо другого файла, разделяемого на страницы, например, публикации какой-либо части базы знаний, то sc.n-страницей считается только часть страницы, на которой изображен sc.n-текст, в то время как страница указанного файла может быть больше за счет, например, белых полей по краям страницы, необходимых для последующей распечатки.}
        
        \scnheader{строчка sc.n-текста}
        \scntext{примечание}{Максимальное количество символов в строчках sc.n-текста для каждого sc.n-текста фиксировано и определяется конкретным вариантом размещения sc.n-текста. При этом, в зависимости от отступов в рамках конкретного sc.n-предложения, строчка sc.n-текста может начинаться не с левого края sc.n-текста (но всегда с какой-то из вертикальных линий разметки) и иметь произвольную длину, ограничиваемую правой границей sc.n-страницы.}
        
        \scnheader{линия разметки sc.n-текста}
        \scnidtf{sc.n-text marking line}
        \scnidtf{табуляционная линия sc.n-текста}
        \scnidtf{вертикальная линия разметки sc.n-текста}
        \scnidtf{вертикальная табуляционная линия}
        \scnidtf{вертикальная линия, используемая для упрощения восприятия sc.n-текстов и показывающая уровень отступа для компонентов sc.n-предложений}
        \scntext{пояснение}{1-я линия разметки ограничивает левый край sc.n-страницы, 2-я линия разметки располагается примерно между 5 и 6 символами строчки и т.д. Расстояние между линиями разметки может меняться в зависимости от размера шрифта, однако в рамках одного sc.n-текста всегда остается одинаковым. Общее количество линий разметки ограничивается максимально возможной шириной sc.n-страницы в конкретном файле ostis-системы, содержащем данный sc.n-текст.}
        
        \scnheader{следует отличать*}
        \begin{scnhaselementset}
            \scnitem{страница sc.n-текста}
            \scnitem{строчка sc.n-текста}
            \scnitem{строка}
        \end{scnhaselementset}

        \bigskip

        \begin{scnset}
            \scnitem{SCn-код}
            \scnitem{SCs-код}
        \end{scnset}
        \begin{scnrelfromset}{сходство}
            \scnitem{Алфавит SCs-кода}
            \begin{scnindent}
                \begin{scnreltolist}{алфавит}
                    \scnitem{SCs-код}
                    \scnitem{SCn-код}
                \end{scnreltolist}
            \end{scnindent}
        \end{scnrelfromset}
        \begin{scnindent}
            \scnrelboth{семантически эквивалентная информационная конструкция}{\scnfilelong{Алфавит символов \textit{SCs-кода} является также алфавитом символов и \textit{SCn-кода}, т.е. \textit{алфавиты}* этих языков совпадают.}}
        \end{scnindent}
        \scntext{сходство}{Все компоненты sc.s-текстов используются также и в sc.n-текстах:
            \begin{scnitemize}
                \item sc-идентификаторы
                \item sc.s-коннекторы
                \item модификаторы sc.s-коннекторов с соответствующими разделителями (двоеточиями)
                \item разделители, используемые в sc-выражениях, обозначающих sc-множества, заданные перечислением элементов с соответствующими разделителями (\textit{точкой с запятой} или \textit{круглым маркером})
                \item \textit{круглые маркеры} в перечислениях идентификаторов \mbox{sc-элементов}, связанных однотипными sc-коннекторами с однотипными модификаторами с заданным sc-элементом
                \item разделители предложений (двойные точки с запятой) (опускаются при преобразовании \mbox{sc.s-предложений} в \mbox{sc.n-предложения})
                \item ограничители присоединенных sc.s-предложений (опускаются при преобразовании sc.s-предложений в sc.n-предложения).
            \end{scnitemize}
        }
        \scntext{отличие}{В отличие от sc.s-текстов в sc.n-текстах:
            \begin{scnitemize}
                \item добавляются новые виды sc-выражений (а именно --- sc-выражений, имеющих двухмерный характер);
                \item добавляется новый вид разделителей предложений --- пустая строчка;
                \item меняется размещение предложений с учетом двухмерного характера такого размещения.
            \end{scnitemize}
        }
        \scntext{отличие}{В \textit{SCn-коде} по сравнению с \textit{SCs-кодом} добавляются новые виды \textit{sc-выражений}:
            \begin{scnitemize}
                \item \textit{sc-выражение}, представляющее собой двухмерный \textit{\mbox{sc.n-текст}}, ограниченный \textit{sc.n-контуром} или \textit{sc.n-рамкой}. Каждый \textit{sc.n-контур} изображается условно в виде \textit{открывающей фигурной скобки} и расположенной строго \uline{под} ней через несколько строчек \textit{закрывающей фигурной скобки}. Внутри указанных скобок (начиная от линии вертикальной разметки, на которой расположены сами скобки, и до правого края \textit{страницы}) размещается sc.n-текст. Полученный sc.n-контур является изображением структуры, являющейся результатом трансляции указанного sc.n-текста в SC-код. Каждая \textit{sc.n-рамка} изображается аналогичным образом, только вместо \textit{фигурных скобок} в ней используются \textit{квадратные скобки}, либо \textit{квадратные скобки} с \textit{восклицательным знаком} (в случае файла-образца);
                \item \textit{sc-выражение}, представляющее собой двухмерный \textit{sc.g-текст}, ограниченный \textit{\mbox{sc.n-контуром}} или \textit{\mbox{sc.n-рамкой}}.
                \item \textit{sc-выражение}, представляющее собой ограниченное \textit{sc.n-рамкой} двухмерное графическое изображение \textit{информационной конструкции}, закодированной в некотором \textit{файле ostis-системы}. Такой \textit{информационной конструкцией} может быть таблица, рисунок, фотография, диаграмма, график и многое другое.
            \end{scnitemize}
        }
        \begin{scnindent}
            \scntext{примечание}{Нетрудно заметить, что \textit{sc.n-контур} является, по сути, двухмерным эквивалентом \textit{sc-выражения структуры}, а \textit{sc.n-рамка} --- двухмерным эквивалентом \textit{sc-выражения внутреннего файла \mbox{ostis-системы}} или \textit{sc-выражения, обозначающего файл-образец ostis-системы}.}
        \end{scnindent}
            
        \scnheader{sc.n-рамка}
        \scnidtf{sc.n-frame}
        \scnidtf{ограничитель изображения файла \uline{ostis-системы}, используемый в \uline{sc.n-предложениях}}
        \scntext{примечание}{С формальной точки зрения \textit{sc.n-рамка} всегда представляет собой \uline{одну} \textit{строчку sc.n-текста}. Это означает, что \textit{sc.n-рамка} не может быть синтаксически разделена на части в рамках того \textit{sc.n-текста}, в котором она используется, и внутрь нее не могут вставляться, например, \textit{присоединенные sc.n-предложения} или какой-либо другой текст (за исключением случаев, когда \textit{sc.n-рамка} содержит \textit{sc.n-текст}, но в этом случае указанный \textit{sc.n-текст} все равно будет рассматриваться как целостный внешний файл, а не как фрагмент окружающего его \textit{sc.n-текста}).}
        \scntext{примечание}{Обозначается с помощью квадратных скобок: \scnqqi{\[ \scnqqi{,} \]}.}
        
        \scnheader{sc.n-контур}
        \scnidtf{sc.n-contour}
        \scnidtf{используемый в \uline{sc.n-предложениях} ограничитель, являющийся изображением структуры}

        \bigskip

        \begin{scnset}
            \scnitem{sc.s-предложение}
            \scnitem{sc.n-предложение}
        \end{scnset}
        \scntext{сходство}{Понятие \textit{sc.n-предложения} является естественным обобщением понятия \textit{sc.s-предложения}. Более того, \uline{аналогичным} для \textit{sc.s-предложений} образом вводятся понятия:
            \begin{scnitemize}
                \item \textit{простого sc.n-предложения}
                \item \textit{сложного sc.n-предложения}
                \item \textit{sc.n-предложения, содержащего присоединенные sc.n-предложения}
                \item \textit{sc.n-предложения, не содержащего присоединенные sc.n-предложения}
                \item \textit{присоединенного sc.n-предложения}
                \item \textit{неприсоединенного sc.n-предложения}
            \end{scnitemize}
        }
        \begin{scnrelfromlist}{отличие}
            \scnfileitem{\uline{Если} каждое \textit{неприсоединенное sc.s-предложение} \uline{либо} являетcя первым предложением \textit{sc.s-текста}, \uline{либо} начинается после \textit{разделителя sc.s-предложений} (\textit{двойной точки с запятой}), \uline{то} каждое \textit{неприсоединенное sc.n-предложение} начинается с начала новой строчки}
            \scnfileitem{\uline{Если} каждое \textit{присоединенное sc.s-предложение} начинается либо после открывающего ограничителя присоединенных sc.s-предложений (открывающей круглой скобки со звездочкой), \uline{либо} после \textit{разделителя sc.s-предложений}, \uline{то} каждое \textit{присоединенное sc.n-предложение} начинается с новой строчки под sc-идентификатором, которым завершается то sc.n-предложение (и соответственно, sc.s-предложение), в которое встраивается данное \textit{присоединенное sc.n-предложение}}
            \scnfileitem{Первый \textit{sc-идентификатор}, входящий в состав \textit{sc.n-предложения} до \textit{sc.s-коннектора} выделяется \uline{жирным} курсивом}
            \scnfileitem{В \textit{sc.n-предложениях двойная точка с запятой} не используется в качестве признака завершения этих предложений и, соответственно, не используется в качестве разделителя \textit{sc.n-предложений}. Таким разделителем является \textit{пустая строчка}.}
        \end{scnrelfromlist}
        \scntext{отличие}{Благодаря двухмерности SCn-кода появляются более широкие возможности (степени свободы) для наглядного и компактного размещения sc.n-предложений.}
        \begin{scnindent}
            \begin{scnrelfromlist}{уточнение}
                \scnfileitem{При оформлении sc.n-предложения осуществляется четкая \uline{табуляция} всех присоединенных к нему sc.n-предложений, присоединяемых к исходному по вертикали. Вертикальная линия табуляции задает левую границу исходного (максимального) sc.n-предложения или левую границу присоединенного sc.n-предложения, присоединяемого по вертикали. Левая граница sc.n-предложения задает начало первого sc-идентификатора, входящего в состав этого sc.n-предложения, а также начало sc.s-коннектора, инцидентного указанному sc-идентификатору и размещаемого \uline{строго под} этим sc-идентификатором. Расстояние между вертикальными табуляционными линиями фиксировано и примерно равно максимальной длине sc.s-коннектора.}
                \scnfileitem{Разделителями \textit{sc.n-предложений} (как и \textit{sc.s-предложений}) являются двойные точки с запятой. Этот же разделитель отделяет заголовок \textit{sc.n-статьи} от первого предложения этой \textit{sc.n-статьи}. При этом заголовок \textit{sc.n-статьи} можно трактовать как вырожденное \textit{sc.n-предложение}, состоящее только из одного идентификатора.}
                \scnfileitem{В отличие от sc.s-текстов: в sc.n-текстах sc.s-коннектор может быть инцидентен предшествующему sc-идентификатору (как простому, так и sc-выражению) не только \scnqq{по горизонтали}, но и по вертикали. Для этого sc.s-коннектор размещается строго \uline{под} предшествующим ему sc-идентификатором.}
                \scnfileitem{Кроме того \scnqq{по вертикали} sc-идентификатор может быть инцидентен не одному, а \uline{нескольким} sc.s-коннекторам, которые последовательно \scnqq{по вертикали} размещаются \uline{под} указанным sc-идентификатором. Это позволяет в рамках одного sc.n-предложения представлять произвольное число \scnqq{ответвлений} от каждого sc-идентификатора, т.е. произвольное число sc.s-коннекторов, инцидентных этому sc-идентификатору.}
                \scnfileitem{Каждый sc-идентификатор, включая sc-выражение, ограничиваемого фигурными или квадратными скобками, должен размещаться сразу правее вертикальной разметочной линии, если \uline{под ним} размещается sc.s-коннектор.}
                \scnfileitem{Каждый sc.s-коннектор выделяется жирным некурсивным шрифтом и, если он находится \uline{под} инцидентным ему sc-идентификатором, размещается строго между двумя вертикальными разметочными линиями, прижимаясь при этом к левой из этих двух разметочных линий.}
            \end{scnrelfromlist}
        \end{scnindent}

        \scnheader{SCn-код}
        \scntext{правило синтаксической трансформации}{Поскольку по отношению к SCn-коду SCs-код является \textit{синтаксическим ядром языка*}, SCn-код можно рассматривать как результат интеграции нескольких направлений расширения SCs-кода, в основе которых лежат правила синтаксической трансформации sc.s-текстов и sc.n-текстов, ориентированные на повышение эффективности использования тех возможностей обеспечения наглядности и компактности sc.n-текстов, которые открываются при переходе от линейности sc.s-текстов к двухмерности sc.g-текстов}
        
        \scnheader{sc.n-предложение}
        \scnidtf{sc.n-sentence}
        \begin{scnrelfromlist}{заданная операция}
            \scnitem{операция преобразования sc.s-предложения в sc.n-предложение*}
            \begin{scnindent}
                \scnsubset{синтаксическая трансформация*}
                \scntext{пояснение}{Каждое \textit{sc.s-предложение}, записываемое линейно (горизонтально) может быть преобразовано в соответствующее двухмерное \textit{sc.n-предложение}. Перечислим основные правила трансформации sc.s-предложений в sc.n-предложения:
                    \begin{scnitemize}
                        \item sc.s-коннектор может размещаться на следующей строчке под предшествующим \textit{sc-идентификатором}, начиная с того же символа следующей строчки, что и указанный sc-идентификатор;
                        \item если sc-идентификатор переносится на следующую строчку, то его продолжение на следующей строчке осуществляется с таким же отступом от начала строчки, с каким указанный sc-идентификатор начинается;
                        \item перечисление sc-идентификаторов, разделенных точкой с запятой, может осуществляться не в строчку, а в столбик при размещении каждого следующего sc-идентификатора строго под предшествующим. При этом, разделительная точка с запятой может быть заменена \textit{круглым маркером}, который помещается \uline{перед} каждым перечисляемым \mbox{sc-идентификатором};
                        \item закрывающая фигурная или квадратная скобка может быть размещена строго \uline{под} соответствующей открывающей скобкой;
                        \item sc-идентификатор в sc.n-предложении может быть связан с другими sc-идентификаторами через несколько разных sc.s-коннекторов. При этом, каждый из этих sc.s-коннекторов размещается строго под предшествующим, но только после того, когда будет завершена запись всей, в общем случае разветвленной, цепочки sc.s-коннекторов и sc-идентификаторов, которая начинается с предшествующего sc.s-коннектора. В SCs-коде аналога таким предложениям с неограниченной возможностью описания разветвленных связей sc-идентификаторов нет. Следовательно, если в sc.s-тексте sc-идентификатор может быть инцидентен не более, чем двум sc.s-коннекторам (слева и справа от него), то в sc.n-тексте sc-идентификатор может быть дополнительно инцидентен неограниченному числу (причем, не обязательно одинаковых) sc.s-коннекторов, которые размещаются по вертикали строго под ним.
                    \end{scnitemize}
                }
            \end{scnindent}
            \scnitem{операция присоединения sc.n-предложения*}
            \begin{scnindent}
                \scnsubset{синтаксическая трансформация*}
                \scntext{пояснение}{Некоторое sc.n-предложение может быть присоединено к другому sc.n-предложению, если в этом другом sc.n-предложении есть sc-идентификатор (но не sc.s-модификатор), с которого начинается первое (присоединяемое) sc.n-предложение.Присоединение в происходит следующим образом:
                    \begin{scnitemize}
                        \item начальный sc-идентификатор присоединяемого предложения опускается;
                        \item оставшаяся часть sc.n-предложения, начиная от sc.s-коннектора, записывается под таким же sc-идентификатором, но входящим в состав того sc.n-предложения, к которому присоединяется данное sc.n-предложение. С учетом этого смещаются все отступы в присоединяемом sc.n-предложении.
                    \end{scnitemize}
                    Таким образом может формироваться произвольное число любых разветвлений.
                    }
            \end{scnindent}
        \end{scnrelfromlist}
        \scntext{примечание}{По сути, семантика sc.n-предложения --- множество маршрутов в sc-тексте, возможно пересекающихся и исходящих из заданного sc-элемента}
        
        \begin{scnset}
        	\scnitem{sc.g-текст}
        	\scnitem{sc.s-текст}
        	\scnitem{sc.n-текст}
        \end{scnset}
        \scntext{примечание}{Любой \textit{sc.g-текст} можно легко представить с помощью \textit{sc.s-текста} и \textit{sc.n-текста}.}
        \begin{scnindent}
	        \scnrelfrom{пример}{\scnfileimage[20em]{Contents/part_kb/src/images/scs/scs_text_example.png}}
	        \begin{scnindent}
	        	\scniselement{sc.g-текст}	
		    \end{scnindent}
		    \scnrelfrom{пример}{\scnfileimage[20em]{Contents/part_kb/src/images/sd_lang/example_scs.png}}
		    \begin{scnindent}
		        \scniselement{sc.g-текст}
		    \end{scnindent}
		    \scnrelfrom{пример}{\scnfileimage[20em]{Contents/part_kb/src/images/sd_lang/example_scn.png}}
	        \begin{scnindent}
	        	\scniselement{sc.g-текст}
	        \end{scnindent}
		\end{scnindent}
      	\input{Contents/part_kb/src/intro_lang/intro_latex_lang.tex}

        \bigskip
    \end{scnstruct}
    \scnendcurrentsectioncomment
\end{SCn}
